\subsection*{}

%%%%%%PROBLEMATIQUE %%%%%%%%%%
Nous retenons ainsi de cette présentation des termes du sujet la possibilité, tant technique qu'épistémologique, de mettre en lumière, grâce aux méthodes apportées par les humanités numériques et au travers de l'objet d'étude particulier qu'est la revue, de nouvelles manières de comprendre le triptyque entre un réseau d'acteurs, des enjeux de santé et des territoires d'exercice particuliers que sont les territoires coloniaux. A cet égard, nous avons choisit de travailler sur trois revues qui entretiennent des liens particuliers de par le partage de leurs acteurs, de leurs thématiques et, à priori, de leur terrain. Les Annales d'hygiène et de médecine coloniales (qui deviennent en 1919 les Annales de médecine et de pharmacie coloniales), les Archives de médecine navale (qui deviennent en 1914 les Archives de médecine et de pharmacie navales) et le Bulletin de la Société de pathologie exotique.

Bien connu des chercheurs s’intéressant à ces problématiques, ces trois revues n'ont jamais fait l'objet d'une étude approfondie.  Pour cette année, nous avons souhaité mettre en place un ensemble de matériaux et de méthodes qui nous permettront de mettre en oeuvre comme il se doit nos recherches l'année prochaine et de répondre aux questions de recherche que nous avons évoqué jusqu'ici.

