\section*{Historiographie}

Tant au regard de l'étude des sciences que de l'histoire des sociétés coloniales, les années 1990 ont marqué un tournant historiographique. En réaction au développement du post-colonialisme dans les universités anglo-saxonnes, divers chercheurs\footnote{Tels que \cite{cooper_tensions_1997}} se sont attachés à « repenser le colonialisme »  en considérant une approche socio-historique du fait colonial ainsi qu'en cherchant à percer les relations complexes entre populations colonisées et coloniales. Ce renouvellement s'est aussi matérialisé dans une  l'étude des interactions entre médecine, empires et gouvernement colonial\footnote{\cite{anderson_where_1998,marks_what_1997,arnold_imperial_1989,harrison_towards_1990}}. Une nouvelle historiographie a notamment été marquée par l'intégration de l’histoire de la médecine coloniale dans l’histoire sociale de la médecine, déplaçant le centre de gravité des métropoles vers les mondes « non européens », les périphéries impériales et les espaces transnationaux. 

En France, ce mouvement a été amorcé au tournant des années 2000 par la publication d'un ensemble de thèses et mémoires interrogeant la construction de savoirs en situation coloniale ou bien l'émergence de spécialités dites "coloniales". \footnote{\cite{sibeud_construction_1999,singaravelou_ecole_1999,blais_les_2000,monnais-rousselot_medecine_1999,bretelle-establet_sante_1999,singaravelou_promethee_2013}}. 
Plus récemment, des chercheurs français tels que Guillaume Lachenal\footnote{\cite{lachenal_medecin_2010}} ont étudié l'enjeu particulier du \og laboratoire colonial \fg et de l'expérimentation. En travaillant sur un archétype de \of laboratoire en plein air \fg qu'a été l'expérience d'Eugène Jamot dans le Haut-Nyong Camerounais dans l'entre-deux guerres, il parvient à « estimer l’effet matériel et symbolique des politiques et des discours médicaux »\footnote{\cite{lachenal_medecin_2010}p.123}, s'inscrivant ainsi dans la continuité des études portées par Stoler et Cooper\footnote{\cite{cooper_tensions_1997}} visant à réévaluer les discours officiels à l'aune des réalités du terrain. Ainsi, entre discours et réalités, le rapport au terrain, au territoire est d'autant plus important que les médecins ont eu un rôle majeur dans la maîtrise des espaces par la maîtrise des maladies.  


A partir de 1880, le développement de nouveaux paradigmes scientifiques (théorie des germes, rôle des parasites) ainsi que l’expansion et la stabilisation des domaines coloniaux et l’institutionnalisation des « sciences coloniales » font de la communication scientifique un enjeu majeur de pouvoir tant dans un cadre de compétition que de coopération entre États. Dès lors, la communication scientifique entre métropole et colonies mais aussi entre les colonies et avec les autres empires coloniaux prend une place de premier ordre. Dans cette mesure, nous nous sommes intéressés à la construction et à la circulation des réseaux de savoirs en une période où les relations scientifiques se structurent autour de congrès internationaux, revues et sociétés savantes dans un contexte d'accélération générale des circulations tant humaines que matérielles et financières.

 %%%REVUES MEDICALES

 \subsubsection*{Appréhender l'objet revue médicale}
Les revues scientifiques sont, à cet égard, une source particulièrement riche d'information. Développées à la fin du XVIIIe et au début du XIXe siècle, l'émergence de la revue résulte du développement du modèle des Sociétés savantes comme lieux de savoir, de socialisation et de positionnement scientifique. Depuis les années 1970, la sociologie des sciences a mis en avant "le rôle des revues dans l'émergence des communautés scientifiques et dans la formation des disciplines "\footnote{\cite{duclert_les_2002},p.239}. Leur émergence a aussi été conditionnée par l'accélération des échanges et l'abaissement des prix d'envois. 
Leur histoire est donc avant toute chose celle d'une \og mécanisation et marchandisation du régime épistolaire par le développement de l’imprimerie et des maisons d'édition\fg \footnote{\cite{guedon_lhistoire_2017,tesniere_au_2021}}. Remplaçant les correspondances interpersonnelles, l'objet revue en contribuant à la constitution de réseaux scientifiques plus larges contribue ainsi non seulement à créer des normes linguistiques et argumentatives d'un champ mais aussi, par leur démultiplication numérique depuis 1880, à accroître les spécialisations. L'industrialisation de l’imprimerie et la modernisation des transports au XIXe siècle bouleversent alors les conditions de constitution et de transmission des connaissances. Un des intérêt de l'analyse de la production éditoriale pour nous est donc de saisir « l’intrication complexe de la médecine généraliste et des spécialités avec les préoccupations de santé publique et de formation professionnelle »\footnote{\cite{tesniere_les_2014}}. 


\subsubsection{Quelle discipline pour la médecine en situation coloniale ? }
Définir ce que recouvrent les dénominations de médecine "coloniales", "tropicales", "exotiques", "navale", peut être appréhendé au travers une étude rapprochée des revues qui se présentent comme relevant de ces "spécialisations" en étudiant précisément comment elles se différencient par les thématiques abordées mais aussi par les références employées et les acteurs qui y prennent part. Plusieurs historiens ont tenté de différencier ces dénominations de spécialités médicales : en fonction de la formation des acteurs qui s'en revendiquent\footnote{\cite{cook_rise_2014,braesco_former_2017,osborne_emergence_2014}}, au regard des réseaux d'acteurs formés en \og Communauté épistémique \fg\footnote{\cite{haas_introduction_nodate,packard_networks_2012}}, par les discours et utilisations politiques de cette médecine \footnote{\cite{marks_what_1997}} ou en menant une étude sur un terrain précis\footnote{\cite{anderson_colonial_2006,}}.

La thèse de Marie-Albane de Suremain sur \textit{L'Afrique des revues}\footnote{\cite{suremain_de_afrique_2001}} fait office d'exemple pour notre approche en tant qu'elle parvient à interroger, à partir des revues de sciences sociales, l'évolution des acteurs et des structures produisant des savoirs estampillés comme "coloniaux". Cette démarche fait particulièrement écho à ce que nous souhaitons engager pour ce mémoire en tant qu'exemple d'une recherche liant thématiques, espaces et acteurs engagés dans la production de savoirs en situation coloniale, démontrant, au fil du premier vingtième siècle l'institutionnalisation de la production des sciences sociales dites "coloniales". Si cette étude a été menée sur les revues de sciences humaines et sociales, il n'en existe pas à notre connaissance sur les revues médicales. 

L'étude des revues et plus particulièrement des revues médicales est un lieu commun en histoire des sciences. Mais l'étude plus précise de l'histoire de leur édition reste relativement récente. En effet, l'intérêt des historiens pour la revue scientifique a longtemps été (et est toujours dans une large mesure) au niveau des contributeurs de revues scientifiques\footnote{\cite{csiszar_scientific_2018}} tandis que Valérie Tesniere propose une lecture "terre à terre" de la revue. Suivant les recherches en histoire du livre et de l'édition \footnote{\cite{mollier_argent_1988} et l'Histoire de l'édition française de Roger Chartier et Henri-Jean Martin.}. L'existence de sociétés de recherche dédiées aux questions des revues telles que la  \textit{Research Society for American Periodicals} et la \textit{European Society for Periodical Research} attestent de la vivacité de ce champ de recherche qui s'est trouvé entre autres bouleversé par la numérisation massive des revues. Les démarches d'ouverture des bibliothèques et archives en ligne ont alors permis et accompagné le développement de nouvelles heuristiques grâce à des méthodes de codage et d'encodage des revues que nous envisageons dans le cadre du master humanités numériques de l'université Paris Sciences et Lettres. 




