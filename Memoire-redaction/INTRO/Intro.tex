\mainmatter
\chapter*{Introduction}

\subsubsection{Revues en situations coloniales}
%% ACCROCHE%% 
Apanage du « Prométhée colonial », la médecine est, avec l’éducation, au cœur des discours de légitimation de la colonisation au nom de la « mission civilisatrice » des États colonisateurs. Or, si les sciences de la nature ont fait l'objet de nombreuses sollicitations par les autorités coloniales pour la \og mise en valeur des colonies\fg, il est nécessaire de restituer les stratégies institutionnelles des réseaux scientifiques à l'aune des bénéfices épistémologiques de l'inscription de leurs recherches dans un cadre colonial\footnote{\cite{bonneuil_savants_1991}}.
Dans cette mesure, en nous interrogeant sur les conditions matérielles, sociales et géographique de ces ces réseaux scientifiques, l'objet revue s'est imposé comme une source riche. Les revues, en tant que \og journaux professionnels \fg \footnote{Jean Delhombre, cité par \cite{tesniere_au_2021}} sont alors des supports majeurs pour saisir l'évolution de champs scientifiques dédiés. 

Ces considérations prennent une envergure particulière dans le cadre des disciplines médicales en situation coloniale. En effet, si les disciplines médicales se sont très tôt munies de revues\footnote{\cite{tesniere_les_2014}}, l’importante distance physique des médecins en situation coloniale, leur confrontation à des milieux pathogènes peu connus et les enjeux politiques dont ils avaient la charge (notamment assurer la santé des européens puis celle des ouvriers et enfin des populations locales) firent des revues médicales un objet clé de la colonisation.  Développées dans l’ensemble des pays colonisateurs, les revues médicales dédiées aux situations coloniales ont permis à la fois de mettre en valeur les avancées de chaque pays dans une optique de compétition mais aussi de permettre une coopération transimpériale sur des thématiques communes, difficilement retenues par des frontières politiques.

%%%%%HISTORIOGRAPHIE%%%%%

\section*{Historiographie}

Tant au regard de l'étude des sciences que de l'histoire des sociétés coloniales, les années 1990 ont marqué un tournant historiographique. En réaction au développement du post-colonialisme dans les universités anglo-saxonnes, divers chercheurs\footnote{Tels que \cite{cooper_tensions_1997}} se sont attachés à « repenser le colonialisme »  en considérant une approche socio-historique du fait colonial ainsi qu'en cherchant à percer les relations complexes entre populations colonisées et coloniales. Ce renouvellement s'est aussi matérialisé dans une  l'étude des interactions entre médecine, empires et gouvernement colonial\footnote{\cite{anderson_where_1998,marks_what_1997,arnold_imperial_1989,harrison_towards_1990}}. Une nouvelle historiographie a notamment été marquée par l'intégration de l’histoire de la médecine coloniale dans l’histoire sociale de la médecine, déplaçant le centre de gravité des métropoles vers les mondes « non européens », les périphéries impériales et les espaces transnationaux. 

En France, ce mouvement a été amorcé au tournant des années 2000 par la publication d'un ensemble de thèses et mémoires interrogeant la construction de savoirs en situation coloniale ou bien l'émergence de spécialités dites "coloniales". \footnote{\cite{sibeud_construction_1999,singaravelou_ecole_1999,blais_les_2000,monnais-rousselot_medecine_1999,bretelle-establet_sante_1999,singaravelou_promethee_2013}}. 
Plus récemment, des chercheurs français tels que Guillaume Lachenal\footnote{\cite{lachenal_medecin_2010}} ont étudié l'enjeu particulier du \og laboratoire colonial \fg et de l'expérimentation. En travaillant sur un archétype de \of laboratoire en plein air \fg qu'a été l'expérience d'Eugène Jamot dans le Haut-Nyong Camerounais dans l'entre-deux guerres, il parvient à « estimer l’effet matériel et symbolique des politiques et des discours médicaux »\footnote{\cite{lachenal_medecin_2010}p.123}, s'inscrivant ainsi dans la continuité des études portées par Stoler et Cooper\footnote{\cite{cooper_tensions_1997}} visant à réévaluer les discours officiels à l'aune des réalités du terrain. Ainsi, entre discours et réalités, le rapport au terrain, au territoire est d'autant plus important que les médecins ont eu un rôle majeur dans la maîtrise des espaces par la maîtrise des maladies.  


A partir de 1880, le développement de nouveaux paradigmes scientifiques (théorie des germes, rôle des parasites) ainsi que l’expansion et la stabilisation des domaines coloniaux et l’institutionnalisation des « sciences coloniales » font de la communication scientifique un enjeu majeur de pouvoir tant dans un cadre de compétition que de coopération entre États. Dès lors, la communication scientifique entre métropole et colonies mais aussi entre les colonies et avec les autres empires coloniaux prend une place de premier ordre. Dans cette mesure, nous nous sommes intéressés à la construction et à la circulation des réseaux de savoirs en une période où les relations scientifiques se structurent autour de congrès internationaux, revues et sociétés savantes dans un contexte d'accélération générale des circulations tant humaines que matérielles et financières.

 %%%REVUES MEDICALES

 \subsubsection*{Appréhender l'objet revue médicale}
Les revues scientifiques sont, à cet égard, une source particulièrement riche d'information. Développées à la fin du XVIIIe et au début du XIXe siècle, l'émergence de la revue résulte du développement du modèle des Sociétés savantes comme lieux de savoir, de socialisation et de positionnement scientifique. Depuis les années 1970, la sociologie des sciences a mis en avant "le rôle des revues dans l'émergence des communautés scientifiques et dans la formation des disciplines "\footnote{\cite{duclert_les_2002},p.239}. Leur émergence a aussi été conditionnée par l'accélération des échanges et l'abaissement des prix d'envois. 
Leur histoire est donc avant toute chose celle d'une \og mécanisation et marchandisation du régime épistolaire par le développement de l’imprimerie et des maisons d'édition\fg \footnote{\cite{guedon_lhistoire_2017,tesniere_au_2021}}. Remplaçant les correspondances interpersonnelles, l'objet revue en contribuant à la constitution de réseaux scientifiques plus larges contribue ainsi non seulement à créer des normes linguistiques et argumentatives d'un champ mais aussi, par leur démultiplication numérique depuis 1880, à accroître les spécialisations. L'industrialisation de l’imprimerie et la modernisation des transports au XIXe siècle bouleversent alors les conditions de constitution et de transmission des connaissances. Un des intérêt de l'analyse de la production éditoriale pour nous est donc de saisir « l’intrication complexe de la médecine généraliste et des spécialités avec les préoccupations de santé publique et de formation professionnelle »\footnote{\cite{tesniere_les_2014}}. 


\subsubsection{Quelle discipline pour la médecine en situation coloniale ? }
Définir ce que recouvrent les dénominations de médecine "coloniales", "tropicales", "exotiques", "navale", peut être appréhendé au travers une étude rapprochée des revues qui se présentent comme relevant de ces "spécialisations" en étudiant précisément comment elles se différencient par les thématiques abordées mais aussi par les références employées et les acteurs qui y prennent part. Plusieurs historiens ont tenté de différencier ces dénominations de spécialités médicales : en fonction de la formation des acteurs qui s'en revendiquent\footnote{\cite{cook_rise_2014,braesco_former_2017,osborne_emergence_2014}}, au regard des réseaux d'acteurs formés en \og Communauté épistémique \fg\footnote{\cite{haas_introduction_nodate,packard_networks_2012}}, par les discours et utilisations politiques de cette médecine \footnote{\cite{marks_what_1997}} ou en menant une étude sur un terrain précis\footnote{\cite{anderson_colonial_2006,}}.

La thèse de Marie-Albane de Suremain sur \textit{L'Afrique des revues}\footnote{\cite{suremain_de_afrique_2001}} fait office d'exemple pour notre approche en tant qu'elle parvient à interroger, à partir des revues de sciences sociales, l'évolution des acteurs et des structures produisant des savoirs estampillés comme "coloniaux". Cette démarche fait particulièrement écho à ce que nous souhaitons engager pour ce mémoire en tant qu'exemple d'une recherche liant thématiques, espaces et acteurs engagés dans la production de savoirs en situation coloniale, démontrant, au fil du premier vingtième siècle l'institutionnalisation de la production des sciences sociales dites "coloniales". Si cette étude a été menée sur les revues de sciences humaines et sociales, il n'en existe pas à notre connaissance sur les revues médicales. 

L'étude des revues et plus particulièrement des revues médicales est un lieu commun en histoire des sciences. Mais l'étude plus précise de l'histoire de leur édition reste relativement récente. En effet, l'intérêt des historiens pour la revue scientifique a longtemps été (et est toujours dans une large mesure) au niveau des contributeurs de revues scientifiques\footnote{\cite{csiszar_scientific_2018}} tandis que Valérie Tesniere propose une lecture "terre à terre" de la revue. Suivant les recherches en histoire du livre et de l'édition \footnote{\cite{mollier_argent_1988} et l'Histoire de l'édition française de Roger Chartier et Henri-Jean Martin.}. L'existence de sociétés de recherche dédiées aux questions des revues telles que la  \textit{Research Society for American Periodicals} et la \textit{European Society for Periodical Research} attestent de la vivacité de ce champ de recherche qui s'est trouvé entre autres bouleversé par la numérisation massive des revues. Les démarches d'ouverture des bibliothèques et archives en ligne ont alors permis et accompagné le développement de nouvelles heuristiques grâce à des méthodes de codage et d'encodage des revues que nous envisageons dans le cadre du master humanités numériques de l'université Paris Sciences et Lettres. 






%%%PROBLEMATIQUE%%%%%

\subsection*{}

%%%%%%PROBLEMATIQUE %%%%%%%%%%
Nous retenons ainsi de cette présentation des termes du sujet la possibilité, tant technique qu'épistémologique, de mettre en lumière, grâce aux méthodes apportées par les humanités numériques et au travers de l'objet d'étude particulier qu'est la revue, de nouvelles manières de comprendre le triptyque entre un réseau d'acteurs, des enjeux de santé et des territoires d'exercice particuliers que sont les territoires coloniaux. A cet égard, nous avons choisit de travailler sur trois revues qui entretiennent des liens particuliers de par le partage de leurs acteurs, de leurs thématiques et, à priori, de leur terrain. Les Annales d'hygiène et de médecine coloniales (qui deviennent en 1919 les Annales de médecine et de pharmacie coloniales), les Archives de médecine navale (qui deviennent en 1914 les Archives de médecine et de pharmacie navales) et le Bulletin de la Société de pathologie exotique.

Bien connu des chercheurs s’intéressant à ces problématiques, ces trois revues n'ont jamais fait l'objet d'une étude approfondie.  Pour cette année, nous avons souhaité mettre en place un ensemble de matériaux et de méthodes qui nous permettront de mettre en oeuvre comme il se doit nos recherches l'année prochaine et de répondre aux questions de recherche que nous avons évoqué jusqu'ici.



%%%DESCRIPTION DES SOURCES%%%
\subsection*{Pistes proposées pour l'étude de nos revues}

Les humanités numériques offrent à notre avis une nouvelle heuristique pour aborder des sources aussi denses que les revues. Notre objectif est ici de rendre compte des explorations des sources que nous avons mis en place au fil de cette année particulière et riche en découvertes.
Cet enivrement des nouvelles méthodes d'analyse nous a par moment poussé à envisager de (trop) nombreuses pistes de recherche qui n'ont pas toujours pu aboutir cette année en raison des (aussi trop) nombreux engagement.

nous avons souhaité explorer quelques pistes pouvant répondre à nos multiples interrogations afin de nous procurer les matériaux et méthodes pour une analyse approfondie au cours de l'année prochaine.










