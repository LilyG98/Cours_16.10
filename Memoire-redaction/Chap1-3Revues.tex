\chapter{Trois revues de santé en situation coloniale}

\section*{Du papier à l'écran, un autre goût des archives}

%%%%%% HN HN HN HN HN HN %%%%%%%

Internet s’est aujourd’hui imposé comme la première source d’information. Ainsi, notre découverte des trois revues traitées s'est avant toute chose faite en ligne. Des bibliothèques universitaires
\footnote{https://www.biusante.parisdescartes.fr/histoire/medica, consulté le 20/05/2021} à Internet archive\footnote{https://archive.org/, consulté le 20/05/2021}, nos revues peuvent être caractérisées par une très grande accessibilité numérique. 
Leur numérisation est le fruit du travail engagé dès 2005 par la Bibliothèque interuniversitaire de médecine(BIUS)\footnote{Nous souhaitons à cet égard remercier M.Jean-François Vincent et Mme Solène Coutagne, chefs du service historique de la Bibliothèque interuniversitaire de santé, d'avoir répondu à nos questions et de nous avoir fait parvenir les cahiers des charges de chaque revue que nous étudions afin de saisir comment ce projet de numérisation précoce a été conçu ainsi que comment ces choix initiaux ont déterminé la lecture que nous pouvons en avoir aujourd'hui.}. 

Dans le cadre français, les Annales d’hygiène et de médecine coloniale (AHMC), les Archives de médecine navale (AMN) et le Bulletin de la société de Pathologie exotique (BSPE) ont tenu ce rôle d’organe de communication et de rayonnement des avancées françaises mais un trait de caractère les différencie toutefois grandement. Tandis que les AHMC et AMN dépendent d'un corps de métier définit par et sous l'égide d'administrations gouvernementales, le Bulletin dépend pour sa part d'une société savante reposant d'une part sur la notoriété de ses fondateurs et de ses membres honoraires et d'autre part sur les scientifiques présents sur le terrain. Or toute trois partagent des thématiques et acteurs communs dont nous avons cherché à retracer la généalogie.

En portant notre regard sur les revues, nous cherchons à déceler non pas tant le cheminement de leurs recherches mais bien ce que les diverses institutions ont souhaité \og donner à voir \fg. Dans cette partie nous cherchons donc à mettre en exergue les particularités et point communs de chacune des revues avant d'en déployer les inter-relations. Nous verrons dans un premier temps ce qui lie les acteurs et thématiques des Archives de médecine navale et les Annales d'Hygiène et de médecine coloniales avant de les lier à la Société de Pathologie exotique.\footnote{cf la Figure 3 en annexe} 

%%%%%%%%%%%%%%%%%Deux revues institutionnelles%%%%%%%%%%%%%%%%%%%%%%
\section{Deux revues institutionnelles à la généalogie commune : les Archives de médecine navale et les Annales d'Hygiène et de médecine coloniales}

Parmi les trois revues que nous étudions, les Archives de médecine navale et les Annales d'Hygiène et de médecine coloniale ont la particularité de dépendre de ministères. D’une part le ministère des Colonies et le Corps de santé des Troupes coloniales. D’autre part, celui de la Marine et le corps de santé navale. 
Ces deux ministères sont le produit d'une scission en 1890 du Ministère de la Marine et des Colonies, témoignant d'une démarche d'autonomisation du champ colonial par rapport à la navale dont il dépendait jusqu'alors. Dans ce cadre, les deux revues institutionnelles traitées ont un triple objectif : administratif, corporatiste et scientifique. Les liens entre les deux corps restent néanmoins majeurs. 


En effet, l'accès au Corps du Service de santé des troupes coloniales est avant tout présenté comme une forme de "spécialité" des médecins sortant du corps de la navale. Pour intégrer ce corps, il est nécessaire de passer d'abord par une classe préparatoire au concours de l'Ecole militaire de Santé navale pour ensuite intégrer l'Ecole de santé navale de Bordeaux (créée en 1890). La création en 1905 de l'Ecole d'Application du Service de Santé des Troupes Coloniales à Marseille (Le Pharo) fixe le cadre d'un domaine scientifique\footnote{Pour une analyse détaillée du fonctionnement de l'Ecole du Pharo, voir \cite{braesco_former_2017}}. Selon Camille Braesco, la médecine tropicale devient coloniale dans la mesure où elle a trait à des territoires coloniaux. Mais elle ne se résume pas pour autant à la médecine tropicale puisque la formation même des acteurs met avant tout en exergue la capacité d'adaptation de chacun à des situations très variées et ne repose donc pas tant sur l'acquisition de connaissances théoriques particulièrement plus poussées sur les maladies tropicales ni sur les particularités locales des pays colonisés.



\subsection{Les Archives de médecine navale}
Les Archives de médecine navale sont fondées en 1864 par le ministre de la Marine et des Colonies Chasseloup-Laubat et dirigé par Leroy de Méricourt. 180 ans après la création du Service de Santé navale par Signelay et treize années après la première Conférence sanitaire internationale, la création des Archives de médecine navale est le fruit de plusieurs décennies de tentative de création d'un journal relatif au Service de Santé navale\footnote{Voir Archives de médecine navale 1864, n° 01. - Paris : J.-B. Baillière, p.5.}. Les AMN ont d'abord été éditées chez J.B. Baillière (1864-1881), "libraires de l'académie impériale de médecine" jusqu'à l'établissement de la IIIe République, avant d'être publié chez Octave Doin jusqu'à la fin de la Première Guerre mondiale qui la verra être reprise par l'Imprimerie nationale. Cette évolution témoigne de l'évolution du marché éditorial scientifique et des choix de . En effet, les deux changements de maison d'édition sont marquées, dans un premier temps par la progressive perte de vitesse des éditions Baillière dans le domaine de l'édition de revues (notamment dépassé par les éditions Masson\footnote{\cite{tesniere_au_2021}}, qui gèrent depuis 1908 le Bulletin de la Société de Pathologie exotique ) puis par le changement de direction de la maison d'édition Doin et les difficultés financières suite à la Première guerre mondiale\footnote{Cette partie gagnerait à être approfondie par la consultation des archives de chaque maison d'édition.}. 

A partir de la création de Corps de santé des troupes coloniales, les médecins de la marine ont été majoritairement cantonnés aux ports métropolitains, comme en atteste la Reconnaissance d'entités nommées que nous avons menée sur cette revue puis cartographié à l'aide de QGIS (cf Annexe 3)

\subsection{Les Annales d'hygiène et de médecine coloniale}

\begin{changemargin}{5cm}{0cm} 
« La pathologie exotique offre un vaste champ d’études où il y a encore beaucoup à glaner. Notre domaine colonial s’est tellement étendu dans ces dernières années que presque tout est à faire, au point de vue de la géographie médicale. »\footnote{Alexandre Kermorgant, “Introduction”, Annales d’Hygiène et de médecine coloniales, vol.1,n°1, 1898, p.6.}
\end{changemargin}

En posant \og la pathologie exotique \fg et la \og géographique médicale \fg comme principaux domaines d'étude affilié au corps de santé des troupes coloniales, Alexandre Kermorgant, fondateur de la revue et futur membre fondateur de la Société de pathologie exotique (SPE), d'une part s'inscrit dans la continuité des Archives de médecine navale qui débutait déjà chaque numéro de sa revue par un article de géographie médicale. D'autre part il insiste, par la notion de \og pathologie exotique \fg à l'acception réifiante de l'altérité, qu'elle soit géographique ou pathologique. Pourtant, ce fort déterminisme entre lieux, environnements et effets sur la santé alimentés par les pratiques de topographies médicales a été remis en question à la fin du XIXe et début du XXe siècle par la théorie des germes et la découverte des agents pathogènes. Ainsi, si les "Contributions à la géographie médicale" ouvrent chaque numéro de la revue jusqu'à 1914, cette catégorie disparaît par la suite. La disparition de cette catégorie peut aussi s'expliquer par un changement générationnel des contributeurs à la revue à partir de l'après Seconde guerre mondiale, ainsi que la création de la Société de pathologie exotique en 1908 qui devient la principale référence des Annales comme en témoigne, à partir de 1924, la part importante des références au bulletin de la Société au point de constituer, à partir de 1927 presque une rubrique entière rendant compte des principaux apports des séances de la SPE. 

Cette relation des médecins du corps de santé coloniale avec la microbiologie est notamment marquée par le rôle de ces derniers dans la fondation des Instituts Pasteur d'outre-mer\footnote{Calmette (Saigon), Yersin(Nha-Trang), Simond (Karachi), Roubaud (Brazzaville), Mathis (Hanoi), Girard et Robic (Tananarive), Laigret (Dakar).} ainsi que l'importance de la part de leurs contributions au Bulletin de la Société de pathologie exotique. En effet, le rapport intime des médecins coloniaux au terrain permet à ces derniers de prendre pleinement part à la vie de la société par l'envoi d'échantillons et la rédaction d'articles\footnote{Nous avons effectué une liste à visée prosopographique de l'ensemble des médecins des troupes coloniales membres de la Société de pathologie exotique mais n'avons pas eu le temps cette année de l'exploiter convenablement pour une analyse à présenter dans ce premier rendu.}. 

\section{Le Bulletin de la Société de Pathologie exotique}

\begin{changemargin}{5cm}{0cm} 
« ARTICLE PREMIER. – La Société de pathologie exotique a pour but l’étude des maladies exotiques de l’homme et des animaux ; celle de l’hygiène coloniale et de l’hygiène navale et des mesures sanitaires destinées à empêcher l’extension des épidémies et des épizooties d’origine exotique. »
\end{changemargin}
\begin{flushright}
\small
Statuts de la Société de Pathologie exotique, 1907
\end{flushright}

Créée en 1907, la Société de pathologie exotique (SPE) se définit par un rapport à l’exotique. Tant géographique que pathologique, l’exotique exprime ici une mise à distance d’objets d’étude, accessibles uniquement par le voyage, ainsi qu’une extériorisation de l’Autre en le rapportant à des manifestations extérieures relevant de pathologies supposées inconnues. En dressant l’endiguement des « épidémies et épizooties d’origine exotique » comme objectif ainsi qu’en se revendiquant de l’hygiène coloniale et navale, les fondateurs de la société mettent de facto leurs travaux dans le cadre de l’intérêt des autorités politiques françaises mais aussi européennes. 
De par la notoriété de ses membres français comme étrangers, sa proximité au pouvoir politique et la puissante maison d’édition Masson \& Cie, la Société de pathologie exotique a fait de son Bulletin (BSPE) un support majeur de communication scientifique parmi les nombreux nouveaux titres de périodiques caractérisant la « Belle Époque des revues » 2 . Toujours publiée aujourd’hui sous le même nom et par la même maison d’édition\footnote{Malgré le fusionnement en 2005 de Masson avec Elsevier.}, le BSPE a traversé le XXe siècle tandis que la majorité des titres créés pendant cette période n'ont pas survécu à l'arrêt de la Première guerre mondiale \footnote{\cite{tesniere_les_2014}}/ 

L’objectif premier du BSPE est de donner à lire les activités de la SPE. Aux réunions mensuelles correspondent un numéro du Bulletin, la société rend compte de ses correspondances et échanges avec diverses personnalités ou lecteurs ou encore la retranscription de discussions à l’issue d’une communication. Par ailleurs, comme pour toute société savante, par un processus de sélection tant des membres que des communications et mémoires pouvant être faite au cours des séances de la SPE puis être imprimées. A cet égard, il serait nécessaire de se pencher sur les archives même de la SPE auquel nous n’avons pas eu accès pour raison sanitaire15. Ces dernières devraient nous renseigner sur les discussions internes à la Société pour en éclairer finement le fonctionnement interne et le comparer avec ce qui est transcrit dans la revue.
De plus, le capital symbolique de Laveran, des autres membres fondateurs et de l’inscription de la société sous l’égide de l’Institut Pasteur permettent à la SPE d’avoir l’oreille du gouvernement français en métropole comme aux colonies16. Cette position est d’autant plus renforcée par la création dès 1913 de commission de travail s’adressant directement aux pouvoirs publics : opium, maladie du sommeil, paludisme, fièvre jaune font alors l’objet de recherches spécifiques, attribuées à une équipe de membres de la Société chargée de rendre compte aux autorités de leurs recommandations.
Le Bulletin n’est en effet qu’une traduction lissée des activités d’une société éclatée à travers le monde dont les séances ne donnent à voir que les résultats d’un échantillon d’expériences, ne rendant que partiellement compte de l’élaboration des savoirs. Excès de zèle, exagération auto-promotionnelle ou réalité, les présidents successifs de la SPE ont rappelé presque chaque année dans leur allocution à la Société le nombre important de communications qui leur était envoyé supposant un triage important. Il est alors possible, en considérant ces brèches de reconstituer une archéologie des savoirs en œuvre. Elles donnent aussi à voir les différents degrés d’implication des membres de la société savante et d’établir une cartographie professionnelle et géographique de la vie de la revue.



\section{Provincialiser l’Europe?}

Si les trois revues que nous étudions sont éditées en Métropole avant d'être envoyées à travers le monde, il nous semble qu'elles participent à une forme de provincicalisation de l'Europe. En effet, la répartition des médecins de par le monde et la création de laboratoires, d'Instituts Pasteur, et d'Ecole de médecine dans les colonies par des médecins issus notamment des troupes coloniales attestent d'une progressive décentralisation de la production des savoirs. 

Ainsi par exemple, au lendemain de la Grande Guerre, le BSPE rend aussi compte des séances de sociétés devenues « filiales » de la Société de pathologie exotique ; à savoir la Société médico- chirurgicale de l'Ouest africain (SMOA, de 1922 à 1940) et la Société des sciences médicales de Madagascar (SSMM, de 1928 à 1940). Toutes deux liées à l’Institut Pasteur et à l’école de médecine de la ville où elles sont établies, leur fondation par des médecins des troupes coloniales remonte respectivement à 1919 à Dakar et 1909 à Antananarivo. Leur rattachement à la SPE permet alors de gagner grandement en visibilité via le BSPE. Preuve en est qu’aujourd’hui le Bulletin de la SSMM, publié jusqu’à 1928, par exemple n’a pas été numérisé par la BIUS et n’a donc pas le même statut. Du côté de la SPE, l’intérêt d’une filiale permet de rendre compte de son rayonnement en bâtissant à son tour son propre petit empire.
