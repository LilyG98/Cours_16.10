\chapter{Pistes pour l'année prochaine}

\section{Approfondir notre compréhension des contextes de production de la revue}

La situation sanitaire nous ayant empêché de nous déplacer aux Archives nationales d'Outre-mer (ANOM) ou aux Archives du service de Défense de Toulon, nous n'avons pu développer comme nous l'aurions souhaité une étude approfondie des ressorts de chacune des revues. Ainsi, nous émettons le voeu pour l'année prochaine, si la situation sanitaire le permet, de nous rendre aux archives afin de consulter des documents que nous avons déjà identifié parmi les fonds de chacune de ces archives : 

\section{Océrisation et topic modeling}
Si, grâce à Internet Archive, nous avons accès à une version océrisée de nos textes, cette dernière a toutefois souffert d'une numérisation inégale de nos revues en fonction de l'année ou de l'organisation en charge de la numérisation (qualité de la numérisation, orientation du document etc.). De plus, la présence de nombreuses listes, de tableaux et d'images ont encore compliqué leur océrisation par l'algorithme intégré de Internet Archive. Dans ce cadre, si nous avions envisagé de ré-océriser ces revues, nous n'en avons pas trouvé le temps nécessaire cette année et comptons prendre les trois premières semaines du mois de septembre pour y remédier. Cette démarche nous permettrait en effet d'appliquer à notre corpus une  méthodes de lecture distante de notre texte, très utilisée dans le cadre des humanités numériques\footnote{\cite{yang_topic_2011,callaway_push_2020,broadwell_reading_nodate,dawar_comparing_2019}}. 

Nous envisageons de même, dans ce cadre d'intégrer ces nouvelles océrisations à Internet Archive afin de mettre à profit d'autres chercheurs cette démarche longue qu'est l'entraînement d'un algorithme pour la reconnaissance de caractères optiques.

tout d'abord de la modélisation de sujet (ou \textit{Topic modeling}) que nous appliquerions à l'ensemble des contributions à nos revues (ne prenant donc pas en considération les parties de nos revues relatives à l'administration dont elles dépendent. 

\section{A plus long terme}
Nous objectif à long terme est de pouvoir constituer une base de données relationnelle correspondant au schéma en Annexe (Figure 1).

Cette base de données relationnelles nous permettrait ainsi de répondre aux différentes questions que nous nous posons dans le cadre de notre mémoire, à savoir l'étude de l'évolution d'un réseau et de thématiques au cours du premier XXe siècle. 